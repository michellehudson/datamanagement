\documentclass{tufte-handout}

%\geometry{showframe}% for debugging purposes -- displays the margins

\usepackage{amsmath}

% Set up the images/graphics package
\usepackage{graphicx}
\setkeys{Gin}{width=\linewidth,totalheight=\textheight,keepaspectratio}
\graphicspath{{graphics/}}

\title{Research Data Management 101}
\author[Michelle Hudson, Kristin Bogdan]{Michelle Hudson, Kristin Bogdan}
\date{21 February 2014}  % if the \date{} command is left out, the current date will be used

% The following package makes prettier tables.  We're all about the bling!
\usepackage{booktabs}

% The units package provides nice, non-stacked fractions and better spacing
% for units.
\usepackage{units}

% The fancyvrb package lets us customize the formatting of verbatim
% environments.  We use a slightly smaller font.
\usepackage{fancyvrb}
\fvset{fontsize=\normalsize}

% Small sections of multiple columns
\usepackage{multicol}

% Provides paragraphs of dummy text
\usepackage{lipsum}

% These commands are used to pretty-print LaTeX commands
\newcommand{\doccmd}[1]{\texttt{\textbackslash#1}}% command name -- adds backslash automatically
\newcommand{\docopt}[1]{\ensuremath{\langle}\textrm{\textit{#1}}\ensuremath{\rangle}}% optional command argument
\newcommand{\docarg}[1]{\textrm{\textit{#1}}}% (required) command argument
\newenvironment{docspec}{\begin{quote}\noindent}{\end{quote}}% command specification environment
\newcommand{\docenv}[1]{\textsf{#1}}% environment name
\newcommand{\docpkg}[1]{\texttt{#1}}% package name
\newcommand{\doccls}[1]{\texttt{#1}}% document class name
\newcommand{\docclsopt}[1]{\texttt{#1}}% document class option name

\begin{document}

\maketitle% this prints the handout title, author, and date
 
\paragraph{\href{https://github.com/michellehudson/datamanagement/}{https://github.com/michellehudson/datamanagement/}}\label{httpsgithub.commichellehudsondatamanagement} 

\marginnote {\href{mailto:michelle.hudson@yale.edu}{michelle.hudson@yale.edu}, \href{mailto:kristin.bogdan@yale.edu}{kristin.bogdan@yale.edu}}

\marginnote {\subsection{Helpful guides:}\label{guides}
\href{http://guides.library.yale.edu/datamanagement}{http://guides.library.yale.edu/datamanagement}
\href{http://guides.library.yale.edu/data-statistics}{http://guides.library.yale.edu/data-statistics} 
\href{http://guides.library.yale.edu/sciencedata}{http://guides.library.yale.edu/sciencedata}
\href{http://guides.library.yale.edu/eln}{http://guides.library.yale.edu/eln}
\href{http://csssi.yale.edu/datamanagement}{http://csssi.yale.edu/datamanagement}}

\marginnote {\subsection{Additional help:}\label{additional-help}

\href{http://statlab.stat.yale.edu/workshops/}{CSSSI Workshops}
\newline
\href{http://its.yale.edu/services/research-technologies/high-performance-computing}{High Performance Computing}
\newline
\href{http://guides.library.yale.edu/gis}{Geographic Information Systems}}

\section{Overview:}\label{overview}
Using the DDI data lifecycle model as a guide, this workshop will cover the following
questions: 

\begin{enumerate}
\def\labelenumi{\arabic{enumi}.}
\itemsep1pt\parskip0pt\parsep0pt
\item
  What does this stage of the data lifecycle involve?
\item
  What resources are available for doing it well at Yale (\& elsewhere)?
\item
  What are guidelines for managing data at this stage?
\end{enumerate}


\section{What is research data?}\label{what-is-research-data}

Research data is defined as ``the recorded factual material commonly
accepted in the scientific community as necessary to validate research
findings.'' \footnote {OMB Circular A-110.}

There are four types of research data: \footnote {Research data comes in many formats of information: documents,
spreadsheets, field notebooks, survey responses, audio and video
recordings, images, film, specimens, software code, and can be
structured and stored in a variety of file formats.}

\begin{enumerate}
\def\labelenumi{\arabic{enumi}.}
\itemsep1pt\parskip0pt\parsep0pt
\item
  Observational: captured in real time, usually irreplaceable (sensor readings, telescope images, sample data, surveys).
\item
  Experimental: data from lab equipment, can be reproducible but may be expensive (gene sequences).
\item
  Simulation: data generated from test models (climate models).
\item
  Derived or compiled: reproducible but expensive (data mining, compiled databases).
\end{enumerate}


\section{Why manage research data?}\label{why-manage-research-data}

\subsection{Transparency, integrity, and reproducibility:}\label{transparency-integrity-and-reproducibility}

Managing data and making it accessible by peers decreases the chances of an article being retracted because of falsified or missing data sets.
Reproducibility is a fundamental part of scientific research, and failing to make all the components of a research study available makes reproducibility impossible.

\subsection{Compliance:}\label{compliance}

Data management plans are required by funding agencies, and there is increased expectation that the products of federal funding will be required to be accessible to the public.

\subsection{Personal \& professional benefits:}\label{personal-professional-benefits}

If data is managed within your lab, research group, or simply well-organized for your own use, you will save time, energy, and resources.


\marginnote {\subsection{Guidelines:}\label{guidelines-concept}
\begin{enumerate}
\def\labelenumi{\arabic{enumi}.}
\itemsep1pt\parskip0pt\parsep0pt
\item
  Visit the StatLab before you start your project.
\item
  Consider making a data management plan even if you aren't seeking a grant.
\end{enumerate}}
\section{Study concept}\label{study-concept}

\subsection{Tools \& resources:}\label{what-tools-and-resources-are-available}

\paragraph{\href{https://dmp.cdlib.org/}{DMPTool}:}\label{dmptool}

Yale is a DMPTool partner. Logging in with your Yale ID and password
will give you access to the DMPTool, which will give you an overview of
funder requirements (for various NSF, NIH, and other directorates and
divisions), and walk you through building a data management plan.

\paragraph{\href{http://csssi.yale.edu/dmp}{DMP Consultation
Group}:}\label{dmp-consultation-group}

If you have to submit a DMP as part of a grant proposal and have trouble
using the DMPTool or answering questions you think are critical to the
good management of data, you can contact the DMP Consultation Group for
help.

\paragraph{\href{http://csssi.yale.edu/csssi-statistical-consultants-schedule}{StatLab
consultants}:}\label{statlab-consultants}

Even if you aren't submitting a grant proposal, it's a good idea to come
to the StatLab at the beginning of your project. If you know what
analyses you want to do on your data, the StatLab can make sure you set
out to collect your data correctly.

\marginnote
{\subsection{Guidelines for data collection \& documentation:}\label{guidelines-doc}
\begin{enumerate}
\def\labelenumi{\arabic{enumi}.}
\itemsep1pt\parskip0pt\parsep0pt
\item
  Spreadsheets vs.~databases: see the upcoming workshop on database
  design: 4/18/2014, 1:30 - 3:30 CSSSI.
\item
  Consistency: whatever you do, stick with it.
\item
  Level of detail: decide how much detail you'll need now and in the
  future.
\end{enumerate}}

\section{Data collection \& documentation}\label{data-collection-documentation}

\subsection{Yale-supported resources:}\label{yale-supported}

\begin{itemize}
\item
  \href{http://its.yale.edu/services/collaboration-and-file-sharing/box-yale}{Box}
\item
  \href{http://its.yale.edu/services/research-technologies/elab-notebook/labarchives-faqs}{LabArchives}
\item
  \href{http://its.yale.edu/services/email-and-calendars/eliapps-google-apps-education}{EliApps}
\item
  \href{http://its.yale.edu/services/web-and-application-services/qualtrics-survey-tool}{Qualtrics}
\end{itemize}


\subsection{Additional services \& software:}\label{additional-services-software}

\begin{itemize}
\item
  \href{https://github.com/}{GitHub}: https://github.com/
\item
  \href{https://knb.ecoinformatics.org/morphoportal.jsp}{Morpho}
  https://knb.ecoinformatics.org/morphoportal.jsp 
\item
  \href{http://earthcube.org/}{Earthcube} http://earthcube.org/
\item
  \href{http://www.colectica.com/}{Colectica} http://www.colectica.com/
\end{itemize}

\marginnote {\subsection{Guidelines for data processing \& analysis:}\label{guidelines-processing}
\begin{enumerate}
\def\labelenumi{\arabic{enumi}.}
\itemsep1pt\parskip0pt\parsep0pt
\item
  Visit the StatLab before you start your project.
\item
  Keep track of everything you do and always keep versions of your data
  sets.
\item
  Best practices for working with data during analysis -- folder
  structures, naming conventions, statistical package considerations.
\item
  Back up data in accordance with good practice.
\end{enumerate}}


\section{Data processing \& analysis}\label{data-processing-analysis}

\subsection{Tools \& resources:}\label{what-tools-and-resources-are-available-1}
\begin{itemize}
\itemsep1pt\parskip0pt\parsep0pt
\item
  Stata, SAS, MatLab, R, OpenRefine, Python
\item
  \href{http://www.dataone.org/software_tools_catalog}{DataONE software
  tools catalog}
\end{itemize}


\paragraph{Workflow tools}\label{workflow-tools}

\begin{itemize}
\itemsep1pt\parskip0pt\parsep0pt
\item
  \href{https://kepler-project.org}{Kepler}:
  https://kepler-project.org : open source scientific workflow
  application.
\item
  \href{http://www.vistrails.org/index.php/Main_Page}{VisTrails}:
  http://www.vistrails.org : open source scientific workflow
  application, emphasis on visualization.
\end{itemize}

\marginnote{\subsection{Guidelines for data archiving \& preservation:}\label{guidelines-preservation}

\begin{enumerate}
\def\labelenumi{\arabic{enumi}.}
\itemsep1pt\parskip0pt\parsep0pt
\item
  Doing preservation yourself requires format migration and ensuring
  integrity of files.
\item
  Handing over your data to a repository like ICPSR is possible, and
  will ensure the data is usable over the long-term.
\end{enumerate}}

\section{Data archiving \&
preservation}\label{data-archiving-preservation} 

\subsection{What does this stage
involve?}\label{what-does-this-stage-involve}

Archiving and preserving research data is different from distributing it
or backing it up regularly. Preservation ensures long-term retention of
the data and the necessary migration from format to format that will be
required to keep the data usable over a time period.

\subsection{Tools \& resources:}\label{what-tools-and-resources-are-available-2}

A few projects aim to list all the data repositories available for
submission or for finding research data to reuse, and you can search or
browse by subject: \href{http://databib.org/}{DataBib} and \href{http://www.re3data.org}{re3data}

\marginnote {\subsection{Guidelines for data distribution \& citation:}\label{guidelines-citation}

\begin{enumerate}
\def\labelenumi{\arabic{enumi}.}
\itemsep1pt\parskip0pt\parsep0pt
\item
  Give your data set a title and make it easy to credit you.
\item
  Always cite data that you use as if it were as important as the
  journal articles you cite.
\item
  Look for domain-appropriate distribution channels.
\end{enumerate}}

\section{Data distribution \& citation}\label{data-distribution-citation} 

\subsection{What does this stage
involve?}\label{what-does-this-stage-involve-1}

\subsection{Tools \& resources}\label{what-tools-and-resources-are-available-3}

\paragraph{\href{https://www.datacite.org/}{DataCite}}\label{datacite}
\end{document}
